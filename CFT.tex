\documentclass[dvipdfmx,uplatex,b5paper,9pt]{jsreport}
%%amsthm package for theorem environments
\usepackage{amsthm}
%hyperref package
\usepackage[dvipdfmx]{hyperref}
\usepackage{pxjahyper}
\hypersetup{%
bookmarksnumbered=true,%
colorlinks=true,%
pdftitle={ConformalFieldTheory},%
pdfauthor={}
}
%%amsfonts package for mathematical fonts
\usepackage{amsfonts}
%ascmac package for itembox
\usepackage{ascmac}
%amsmath package for mathematical functions
\usepackage{amsmath}
%cleveref package
\usepackage{cleveref}
%mathtools package
\usepackage{mathtools}
\mathtoolsset{showmanualtags}
\usepackage{autonum}
%braket package for Dirac's braket expression
\usepackage{braket}
%bm package for vector expression
\usepackage{bm}
%for toprule etc
\usepackage{booktabs}
%
\usepackage{here}
\usepackage{mleftright}
%
\usepackage{tikz}
\usetikzlibrary{decorations.markings}
\usepackage{pgfplots}
%
\usepackage{siunitx}
\crefname{equation}{Equation}{Equations}

\theoremstyle{definition}
\newtheorem{defn}{Def.}[section]
\crefname{defn}{Definition}{Definitions}

\theoremstyle{definition}
\newtheorem{prop}{Prop.}[section]
\crefname{prop}{Proposition}{Propositions}

\theoremstyle{definition}
\newtheorem{cor}{Cor.}[section]
\crefname{cor}{Corollary}{Corollaries}

\theoremstyle{definition}
\newtheorem{thm}{Theorem}[section]

\theoremstyle{definition}
\newtheorem{lem}{Lemma}[section]
\crefname{lem}{Lemma}{Lemmas}

\theoremstyle{definition}
\newtheorem{eg}{e.g.}[section]

\theoremstyle{definition}
\newtheorem{prob}{problem}[section]

\theoremstyle{definition}
\newtheorem{post}{Postulate}[section]

\theoremstyle{definition}
\newtheorem{fact}{Fact}[section]
\crefname{fact}{Fact}{Facts}
\crefname{equation}{Equation}{Equations}

%\abs & \norm
\DeclarePairedDelimiter\abs{\lvert}{\rvert}%
\DeclarePairedDelimiter\norm{\lVert}{\rVert}%

\title{Conformal Field Theory}
\author{naok1} %適宜自分の名前を書く
\begin{document}
\maketitle
\tableofcontents
\chapter{共形変換と共形代数}
\section{直交変換}
\(d\)次元ユークリッド空間\(\mathbb{R}^{d}\)を考える.\(\mathbb{R}^{d}\)
は平坦な計量\(g_{\mu\nu}=\delta_{\mu\nu}\)を備えている.座標変換
\(x\longrightarrow{}x'\)に対して,計量が作る内積は不変であり,
\begin{align}
	dx^{\mu}g_{\mu\nu}dx^{\nu}=dx'^{\mu}g'_{\mu\nu}dx'^{\nu}
\end{align}
が成り立つ.ここで,
\begin{align}
	dx^{\mu}=\frac{\partial{}x^{\mu}}{\partial{}x'^{\nu}}dx'^{\nu}
\end{align}
を用いれば,
\begin{align}
	dx'^{\mu}g'_{\mu\nu}dx'^{\nu}=\frac{\partial{}x^{\rho}}{\partial{}x'^{\mu}}dx'^{\rho}g_{\rho\sigma}\frac{\partial{}x^{\sigma}}{\partial{}x'^{\nu}}dx'^{\nu}.
\end{align}
従って計量は
\begin{align}
	g_{\mu\nu}\longrightarrow{}g'_{\mu\nu}=\frac{\partial{}x^{\rho}}{\partial{}x'^{\mu}}\frac{\partial{}x^{\sigma}}{\partial{}x'^{\nu}}g_{\rho\sigma}
\end{align}
という変換を受ける.
\begin{defn}[直交変換]
	座標の線型変換,
	\begin{align}
		x^{\mu}\longrightarrow{}\Lambda^{\mu}_{\ \nu}x^{\nu}
	\end{align}
	であってユークリッド計量を変化させないものを直交変換という.
\end{defn}
直交変換では
\begin{align}
	g_{\mu\nu}\Lambda^{\mu}_{\ \rho}\Lambda^{\nu}_{\ \sigma}=g_{\rho\sigma}
\end{align}
が成り立つ.
\begin{prop}
	直交変換ではユークリッド内積が不変.
\end{prop}
\section{共形変換}
\begin{defn}[共形変換]
	座標変換\(x\longrightarrow{}x'\)について,計量が正数倍だけ変換される,すなわち
	\begin{align}
		g_{\mu\nu}\longrightarrow{}\Omega^{2}(x)g_{\mu\nu}
	\end{align}
	となる変換を共形変換という.ここで\(\Omega(x)\)をスケール因子という.
\end{defn}
\begin{prop}
	共形変換では接ベクトルの交わる角度は変わらない.
\end{prop}
\begin{proof}
	点\(x\)において,接ベクトル\(a\),\(b\)を考える.\(a\),\(b\)のなす余弦は
	\begin{align}
		\cos\theta=\frac{g_{\mu\nu}a^{\mu}b^{\nu}}{\sqrt{g_{\mu\mu}g_{\nu\nu}a^{\mu}a^{\mu}b^{\nu}b^{\nu}}}
	\end{align}
	で定義されるが,共形変換\(g_{\mu\nu}\longrightarrow{}\Omega^{2}(x)g_{\mu\nu}\)
	に対して,
	\begin{align}
		\cos\theta\longrightarrow{}\frac{\Omega^{2}(x)g_{\mu\nu}a^{\mu}b^{\nu}}{\Omega^{2}(x)\sqrt{g_{\mu\mu}g_{\nu\nu}a^{\mu}a^{\mu}b^{\nu}b^{\nu}}}=\cos\theta
	\end{align}
	となりこれは不変.従って\(a\),\(b\)のなす角度も不変.
\end{proof}
\begin{prop}
	共形変換について,接ベクトルの大きさはスケール因子を\(\Omega(x)\)として
	\(\abs{\Omega(x)}\)倍だけ変化する.
\end{prop}
\begin{proof}
	\(a\)を点\(x\)における接ベクトルとする.このとき,接ベクトルの
	ノルムの二乗は
	\begin{align}
		g_{\mu\mu}a^{\mu}a^{\mu}\longrightarrow{}\Omega^{2}(x)a^{\mu}a^{\mu}
	\end{align}
	と変化する.従って\(a\)の大きさは\(\abs{\Omega(x)}\)倍だけ変化する.
\end{proof}
\begin{prop}[無限小座標変換]\label{prop::transformation-rule-for-metric}
	無限小座標変換,
	\begin{align}
		x^{\mu}\longrightarrow{}x'^{\mu}=x^{\mu}-\varepsilon^{\mu}(x)
	\end{align}
	を考える.このとき,計量は
	\begin{align}
		g_{\mu\nu}\longrightarrow{}g'_{\mu\nu}=g_{\mu\nu}+\partial_{\mu}\varepsilon_{\nu}(x)+\partial_{\nu}\varepsilon_{\mu}(x).
	\end{align}
	という変換を受ける.
\end{prop}
\begin{proof}
	\begin{align}
		\frac{\partial{}x^{\mu}}{\partial{}x'^{\nu}}=\delta^{\mu}_{\ \nu}+\partial_{\nu}\varepsilon^{\mu}(x)
	\end{align}
	ここで,
	\begin{align}
		\frac{\partial\varepsilon^{\mu}(x)}{\partial{}x'^{\nu}} & =\frac{\partial\varepsilon^{\mu}(x)}{\partial{}x^{\rho}}\frac{\partial{}x^{\rho}}{\partial{}x'^{\nu}}                                                                                                                                   \\
		                                                        & =\frac{\partial\varepsilon^{\mu}(x)}{\partial{}x^{\nu}}+\frac{\partial\varepsilon^{\mu}(x)}{\partial{}x^{\rho}}\frac{\partial{}\varepsilon^{\rho}(x)}{\partial{}x'^{\nu}}\simeq{}\frac{\partial\varepsilon^{\mu}(x)}{\partial{}x^{\nu}}
	\end{align}
	を用いた.従って,計量の変換は
	\begin{align}
		g'_{\mu\nu} & =\frac{\partial{}x^{\rho}}{\partial{}x'^{\mu}}\frac{\partial{}x^{\sigma}}{\partial{}x'^{\nu}}g_{\rho\sigma}                               \\
		            & =(\delta^{\rho}_{\ \mu}+\partial_{\mu}\varepsilon^{\rho}(x))(\delta^{\sigma}_{\ \nu}+\partial_{\nu}\varepsilon^{\sigma}(x))g_{\rho\sigma} \\
		            & =g_{\mu\nu}+\partial_{\mu}\varepsilon_{\nu}(x)+\partial_{\nu}\varepsilon_{\mu}(x).
	\end{align}
	となる.
\end{proof}
\begin{prop}[共形Killing方程式]
	無限小座標変換,
	\begin{align}
		x^{\mu}\longrightarrow{}x'^{\mu}=x^{\mu}-\varepsilon^{\mu}(x)
	\end{align}
	を考える.このとき,この無限小座標変換が共形変換になるためにはある無限小スケール因子\(\omega(x)\)が存在して
	\begin{align}\label{eq::conformal-Killing-equation}
		\partial_{\mu}\varepsilon_{\nu}(x)+\partial_{\nu}\varepsilon_{\mu}(x)=\omega(x)g_{\mu\nu}.
	\end{align}
	となることが必要十分.これを共形Killing方程式 (conformal Killing equation) という.
\end{prop}
\begin{proof}
	無限小変換の計量は\cref{prop::transformation-rule-for-metric}より,
	\begin{align}
		g_{\mu\nu}\longrightarrow{}g'_{\mu\nu}=g_{\mu\nu}+\partial_{\mu}\varepsilon_{\nu}(x)+\partial_{\nu}\varepsilon_{\mu}(x).
	\end{align}
	という変換になる.これが共形変換になるためにはあるスケール因子\(\Omega(x)\)を用いて
	\begin{align}
		g_{\mu\nu}+\partial_{\mu}\varepsilon_{\nu}(x)+\partial_{\nu}\varepsilon_{\mu}(x)=\Omega^{2}(x)g_{\mu\nu}
	\end{align}
	が成り立つことが必要十分.今,ある無限小関数\(\omega(x)\)を
	\begin{align}
		\Omega(x)=1+\frac{\omega(x)}{2}
	\end{align}
	で定義すると,\(\Omega^{2}(x)\simeq{}1+\omega(x)\)となるから,
	\begin{align}
		\partial_{\mu}\varepsilon_{\nu}(x)+\partial_{\nu}\varepsilon_{\mu}(x)=\omega(x)g_{\mu\nu}.
	\end{align}
\end{proof}
\begin{prop}
	\(\varepsilon(x)\)と\(\omega(x)\)は
	\begin{align}\label{eq::infinitesimal-scale-factor-and-conformal-killing-vector}
		\partial_{\mu}\varepsilon^{\mu}(x)=\frac{d}{2}\omega(x),\quad{}\omega(x)=\frac{2}{d}\partial_{\mu}\varepsilon^{\mu}(x).
	\end{align}
	の関係がある.従って共形Killing方程式は
	\begin{align}
		\partial_{\mu}\varepsilon_{\nu}(x)+\partial_{\nu}\varepsilon_{\mu}(x)=\frac{2}{d}g_{\mu\nu}\partial_{\rho}\varepsilon_{\rho}(x).
	\end{align}
	これを満たす\(\varepsilon^{\mu}(x)\)を共形Killingベクトル (conformal Killing vector) という.
\end{prop}
\begin{proof}
	\cref{eq::conformal-Killing-equation}の両辺に\(g^{\mu\nu}\)をかけると,
	\begin{align}
		\partial_{\mu}\varepsilon^{\mu}(x)+\partial_{\mu}\varepsilon^{\mu}(x)=d\omega(x).
	\end{align}
	従って,\(\omega(x)\)は
	\begin{align}
		\partial_{\mu}\varepsilon^{\mu}(x)=\frac{d}{2}\omega(x),\quad{}\omega(x)=\frac{2}{d}\partial_{\mu}\varepsilon^{\mu}(x).
	\end{align}
	で表される.
\end{proof}
共形Killingベクトルの作る共形変換は,
\begin{align}
	g_{\mu\nu}\longrightarrow\mleft(1+\frac{2}{d}\partial_{\mu}\varepsilon^{\mu}(x)\mright)g_{\mu\nu}
\end{align}
となる.
\begin{prop}\label{prop::conformal-Killing-vector-and-omega}
	共形Killingベクトル\(\varepsilon^{\mu}(x)\)について,
	\begin{align}
		\partial^{\mu}\partial_{\mu}\varepsilon_{\rho}(x)=\frac{-d+2}{2}\partial_{\rho}\omega(x).
	\end{align}
\end{prop}
\begin{proof}
	\cref{eq::conformal-Killing-equation}の両辺に左から偏微分演算子\(\partial_{\rho}\)
	を作用させると,
	\begin{align}
		\partial_{\rho}\partial_{\mu}\varepsilon_{\nu}(x)+\partial_{\rho}\partial_{\nu}\varepsilon_{\mu}(x)=\partial_{\rho}\omega(x)g_{\mu\nu}.
	\end{align}
	ここで,\(\mu\),\(\nu\),\(\rho\)について巡回させると,
	\begin{align}
		\partial_{\rho}\partial_{\mu}\varepsilon_{\nu}(x)+\partial_{\rho}\partial_{\nu}\varepsilon_{\mu}(x) & =\partial_{\rho}\omega(x)g_{\mu\nu}, \\
		\partial_{\mu}\partial_{\nu}\varepsilon_{\rho}(x)+\partial_{\mu}\partial_{\rho}\varepsilon_{\nu}(x) & =\partial_{\mu}\omega(x)g_{\nu\rho}, \\
		\partial_{\nu}\partial_{\rho}\varepsilon_{\mu}(x)+\partial_{\nu}\partial_{\mu}\varepsilon_{\rho}(x) & =\partial_{\nu}\omega(x)g_{\rho\mu}.
	\end{align}
	下2つを足して上1つを引くと,
	\begin{align}\label{eq::conformal-formula-1}
		2\partial_{\mu}\partial_{\nu}\varepsilon_{\rho}(x)=-\partial_{\rho}\omega(x)g_{\mu\nu}+\partial_{\mu}\omega(x)g_{\nu\rho}+\partial_{\nu}\omega(x)g_{\rho\mu}
	\end{align}
	この両辺に\(g^{\mu\nu}\)をかけると,
	\begin{align}
		2\partial^{\mu}\partial_{\mu}\varepsilon_{\rho}(x) & =-d\partial_{\rho}\omega(x)+\partial_{\rho}\omega(x)+\partial_{\rho}\omega(x) \\
		                                                   & =(-d+2)\partial_{\rho}\omega(x).
	\end{align}
	従って,
	\begin{align}
		\partial^{\mu}\partial_{\mu}\varepsilon_{\rho}(x)=\frac{-d+2}{2}\partial_{\rho}\omega(x).
	\end{align}
\end{proof}
\begin{prop}
	\begin{align}\label{eq::infinitesimal-scale-factor-d-2}
		\mleft[(d-2)\partial_{\mu}\partial_{\nu}+\partial^{\rho}\partial_{\rho}g_{\mu\nu}\mright]\omega(x)=0.
	\end{align}
	この両辺に\(g^{\mu\nu}\)をかけると,
	\begin{align}\label{eq::infinitesimal-scale-factor-d-1}
		(d-1)\partial^{\mu}\partial_{\mu}\omega(x)=0.
	\end{align}
\end{prop}
\begin{proof}
	共形Killing方程式\cref{eq::conformal-Killing-equation}の両辺に
	\(\partial^{\rho}\partial_{\rho}\)をかけると,\cref{prop::conformal-Killing-vector-and-omega}より,
	\begin{align}
		\frac{1}{2}\mleft[(-d+2)\partial_{\mu}\partial_{\nu}\omega(x)+(-d+2)\partial_{\nu}\partial_{\mu}\omega(x)\mright]=\partial^{\rho}\partial_{\rho}\omega(x)g_{\mu\nu}.
	\end{align}
	従って,
	\begin{align}
		\mleft[(d-2)\partial_{\mu}\partial_{\nu}+\partial^{\rho}\partial_{\rho}g_{\mu\nu}\mright]\omega(x)=0.
	\end{align}
\end{proof}
\begin{enumerate}
	\item \(d=1\)のとき.\(\omega(x)\)は無条件に\cref{eq::infinitesimal-scale-factor-d-1}
	      を満たす.
	\item \(d\geq{}2\)のときは
	      \begin{align}
		      \partial^{\mu}\partial_{\mu}\omega(x)=0
	      \end{align}
	      を満たさなければならない.
	\item \(d\geq{}3\)のとき.
	      \begin{align}
		      \partial^{\mu}\partial_{\mu}\omega(x) & =0, \\
		      \partial_{\mu}\partial_{\nu}\omega(x) & =0
	      \end{align}
	      を満たす必要がある.従って\(3\)次元以上の無限小スケール因子は
	      \(x\)について高々線型で,
	      \begin{align}
		      \omega(x)=A+B_{\mu}x^{\mu}
	      \end{align}
	      と書ける.
\end{enumerate}
\begin{prop}
	\(d\geq{}3\)次元空間の共形Killingベクトル\(\varepsilon_{\mu}(x)\)は
	\begin{align}
		\varepsilon_{\mu}(x)=a_{\mu}+b_{\mu\nu}x^{\nu}+c_{\mu\nu\rho}x^{\nu}x^{\rho}.
	\end{align}
	と表せる.また,\(\varepsilon_{\mu}(x)\)に対応する無限小スケール因子は,
	\begin{align}
		\omega(x)=\frac{2}{d}\left(b^{\mu}_{\ \mu}+2c^{\mu}_{\ \mu\rho}x^{\rho}\right).
	\end{align}
\end{prop}
\begin{proof}
	\(d\geq{}3\)次元空間を考えて,\cref{eq::conformal-formula-1}に
	\begin{align}
		\omega(x)=A+B_{\mu}x^{\mu}
	\end{align}
	を代入すると,
	\begin{align}
		2\partial_{\mu}\partial_{\nu}\varepsilon_{\rho}(x) & =-(\partial_{\rho}B_{\lambda}x^{\lambda})g_{\mu\nu}+\partial_{\mu}(B_{\lambda}x^{\lambda})g_{\nu\rho}+\partial_{\nu}(B_{\lambda}x^{\lambda})g_{\rho\mu} \\
		                                                   & =-B_{\rho}g_{\mu\nu}+B_{\mu}g_{\nu\rho}+B_{\nu}g_{\rho\mu}
	\end{align}
	従って\(\varepsilon_{\mu}(x)\)は\(x^{\mu}\)に関して高々2次で,
	\begin{align}
		\varepsilon_{\mu}(x)=a_{\mu}+b_{\mu\nu}x^{\nu}+c_{\mu\nu\rho}x^{\nu}x^{\rho}.
	\end{align}
	ただし,\(c_{\mu\nu\rho}\)は最後の2つの添字に関して対称で,
	\(c_{\mu\nu\rho}=c_{\mu\rho\nu}\).これを
	\cref{eq::infinitesimal-scale-factor-and-conformal-killing-vector}に代入すると,
	\begin{align}
		\omega(x) & =\frac{2}{d}\partial_{\mu}\left(a^{\mu}+b^{\mu}{\ \nu}x^{\nu}+c^{\mu}_{\ \nu\rho}x^{\nu}x^{\rho}\right)                                                         \\
		          & =\frac{2}{d}\left(b^{\mu}_{\ \nu}\partial_{\mu}x^{\nu}+c^{\mu}_{\ \nu\rho}\partial_{\mu}(x^{\nu}x^{\rho})\right)                                                \\
		          & =\frac{2}{d}\left(b^{\mu}_{\ \nu}\delta^{\nu}_{\ \mu}+c^{\mu}_{\ \nu\rho}(\delta^{\nu}_{\ \mu}x^{\rho}+\delta^{\rho}_{\ \mu}x^{\nu})\right)                     \\\
		          & =\frac{2}{d}\left(b^{\mu}_{\ \mu}+(c^{\mu}_{\ \mu\rho}x^{\rho}+c^{\mu}_{\ \nu\mu}x^{\nu})\right)                                                                \\
		          & =\frac{2}{d}\left(b^{\mu}_{\ \mu}+(c^{\mu}_{\ \mu\rho}+c^{\mu}_{\ \rho\mu})x^{\rho}\right)=\frac{2}{d}\left(b^{\mu}_{\ \mu}+2c^{\mu}_{\ \mu\rho}x^{\rho}\right)
	\end{align}
\end{proof}
\begin{prop}
	次が成り立つ.
	\begin{align}
		b_{\nu\mu}+b_{\mu\nu} & =Ag_{\mu\nu},                                                                       \\
		c_{\mu\nu\rho}        & =\frac{1}{4}\left(-B_{\mu}g_{\nu\rho}+B_{\nu}g_{\rho\mu}+B_{\rho}g_{\mu\nu}\right).
	\end{align}
	すなわち,\(b_{\mu\nu}\)を対称部分と反対称部分
	\begin{align}
		b^{S}_{\mu\nu}=\frac{b_{\mu\nu}+b_{\nu\mu}}{2},\quad{}b^{A}_{\mu\nu}=\frac{b_{\mu\nu}-b_{\nu\mu}}{2}
	\end{align}
	に分けたとき,
	\begin{align}
		b^{S}_{\mu\nu}=\frac{A}{2}g_{\mu\nu}
	\end{align}
	となり,これはスケール変換
	\begin{align}
		{b^{S}}^{\mu}_{\ \nu}=\frac{A}{2}
	\end{align}
	を表す.また,\(c_{\mu\nu\rho}\)に関する変換は
	\begin{align}
		\varepsilon_{\mu}=\frac{1}{4}\left(-B_{\mu}x^{2}+2x_{\mu}B_{\nu}x^{\nu}\right)
	\end{align}
\end{prop}
\begin{proof}
	\cref{eq::conformal-formula-1}に
	\begin{align}
		\varepsilon_{\mu}(x)=a_{\mu}+b_{\mu\nu}x^{\nu}+c_{\mu\nu\rho}x^{\nu}x^{\rho}.
	\end{align}
	を代入すると,
	\begin{align}
		2(c_{\rho\mu\nu}+c_{\rho\nu\mu})=-B_{\rho}g_{\mu\nu}+B_{\mu}g_{\nu\rho}+B_{\nu}g_{\rho\mu}
	\end{align}
	\(c_{\rho\mu\nu}=c_{\rho\nu\mu}\)を用いれば,
	\begin{align}
		c_{\mu\nu\rho}=\frac{1}{4}\left(-B_{\mu}g_{\nu\rho}+B_{\nu}g_{\rho\mu}+B_{\rho}g_{\mu\nu}\right)
	\end{align}
	また,共形Killing方程式\cref{eq::conformal-Killing-equation}に\(\varepsilon_{\mu}\)を
	代入すると,
	\begin{align}
		b_{\nu\mu}+b_{\mu\nu}+2(c_{\mu\nu\rho}+c_{\nu\mu\rho})x^{\rho}=(A+B_{\rho}x^{\rho})g_{\mu\nu}
	\end{align}
	ここで,
	\begin{align}
		c_{\mu\nu\rho}+c_{\nu\mu\rho} & =\frac{1}{4}\left(-B_{\mu}g_{\nu\rho}-B_{\nu}g_{\mu\rho}+B_{\nu}g_{\rho\mu}+B_{\mu}g_{\rho\nu}+B_{\rho}g_{\mu\nu}+B_{\rho}g_{\nu\mu}\right) \\
		                              & =\frac{1}{2}B_{\rho}g_{\mu\nu}
	\end{align}
	であるから,
	\begin{align}
		b_{\nu\mu}+b_{\mu\nu}=Ag_{\mu\nu}.
	\end{align}
\end{proof}
\(d\geq{}3\)次元の共形変換は以下のように分類される.
\begin{enumerate}
	\item 無限小並進変換
	      \begin{align}
		      \varepsilon^{\mu}=a^{\mu}.
	      \end{align}
	\item 無限小回転変換
	      \begin{align}
		      \varepsilon^{\mu}={b^{A}}^{\mu}_{\ \nu}x^{\nu},\quad{}{b^{A}}^{\mu}_{\ \nu}+{b^{A}}^{\nu}_{\ \mu}=0.
	      \end{align}
	\item スケール変換
	      \begin{align}
		      \varepsilon^{\mu}=\frac{A}{2}x^{\mu}.
	      \end{align}
	\item 特殊共形変換 (special conformal transformation)
	      \begin{align}
		      \varepsilon_{\mu}=\frac{1}{4}\left(-B_{\mu}x^{2}+2x_{\mu}B_{\nu}x^{\nu}\right)
	      \end{align}
\end{enumerate}
\chapter{\(d\)次元共形代数}
\section{Lorentz代数}
~\cite{adhara_blog_c31}を参考にした.Minkowski計量を\(\eta\)
としたときLorentz変換は
\begin{align}
	\Lambda^{T}\eta\Lambda=\eta
\end{align}
を満たす線型変換である.
\begin{prop}[Lorentz代数]
	Lorentz変換のLie代数は
	\begin{align}
		\hat{M}_{\mu\nu}\coloneqq{}-ix_{\mu}\frac{\partial}{\partial{}x^{\nu}}+ix_{\nu}\frac{\partial}{\partial{}x^{\mu}}
	\end{align}
\end{prop}
\begin{proof}
	微小Lorentz変換を
	\begin{align}
		\Lambda=1+\varepsilon\simeq{}e^{\varepsilon}
	\end{align}
	で表した時,\(\Lambda^{T}\eta\Lambda=\eta\)に代入すると,
	\begin{align}
		(1+\varepsilon^{T})\eta(1+\varepsilon)=\eta.
	\end{align}
	従って,
	\begin{align}
		(\eta\varepsilon)^{T}+\eta\varepsilon=0.
	\end{align}
	従って\(\varepsilon_{\mu\nu}=\eta_{\mu\rho}\varepsilon^{\rho}_{\ \nu}\)は反対称テンソル.
	また,\(\varepsilon^{\mu\nu}\)も反対称テンソル.ここで,反対称な演算子
	\begin{align}
		\hat{M}_{\mu\nu}\coloneqq{}-ix_{\mu}\frac{\partial}{\partial{}x^{\nu}}+ix_{\nu}\frac{\partial}{\partial{}x^{\mu}}
	\end{align}
	を考えると,これは
	\begin{align}
		\hat{M}_{\mu\nu}x^{\rho} & =i(-x_{\mu}\delta_{\nu\rho}+x_{\nu}\delta_{\mu\rho}).
	\end{align}
	今,微小変換部分
	\begin{align}
		\hat{\varepsilon}:x^{\mu}\mapsto{}(1+\varepsilon)x^{\mu}-x^{\mu}
	\end{align}
	の\(x^{\nu}\)への作用を考えると,
	\begin{align}
		\hat{\varepsilon}x^{\nu} & =\varepsilon^{\mu}_{\ \rho}x^{\rho}\delta^{\nu}_{\rho}                                     \\
		                         & =\left(\varepsilon^{\mu}_{\ \rho}x^{\rho}\frac{\partial}{\partial{}x^{\rho}}\right)x^{\nu} \\
	\end{align}
	従って,
	\begin{align}
		\hat{\varepsilon} & =\left(\varepsilon^{\mu}_{\ \nu}x^{\nu}\frac{\partial}{\partial{}x^{\nu}}\right)=\left(\varepsilon^{\mu\nu}x_{\nu}\frac{\partial}{\partial{}x^{\nu}}\right) \\
		                  & =\frac{1}{2}\left(\varepsilon^{\mu\nu}x_{\nu}\frac{\partial}{\partial{}x^{\nu}}-\varepsilon^{\mu\nu}x_{\mu}\frac{\partial}{\partial{}x^{\mu}}\right)        \\
		                  & =-\frac{i}{2}\varepsilon^{\mu\nu}\left(-ix_{\mu}\frac{\partial}{\partial{}x^{\mu}}+ix_{\nu}\frac{\partial}{\partial{}x^{\nu}}\right)                        \\
		                  & =-\frac{i}{2}\varepsilon^{\mu\nu}\hat{M}_{\mu\nu}.
	\end{align}
\end{proof}
\begin{prop}[Lorentz代数]
	Lorentz変換のLie代数はLorentz代数
	\begin{align}
		[\hat{M}_{\mu\nu},\hat{M}_{\rho\sigma}]=i\eta_{\mu\rho}\hat{M}_{\nu\sigma}-i\eta_{\nu\rho}\hat{M}_{\mu\sigma}-i\eta_{\mu\sigma}\hat{M}_{\nu\rho}+i\eta_{\nu\sigma}\hat{M}_{\mu\rho}
	\end{align}
	を満たす.
\end{prop}
\section{Poincar\'{e}代数}
\begin{prop}
	無限小並進変換のLie代数は
	\begin{align}
		\hat{P}_{\mu}\coloneqq{}\frac{\partial}{\partial{}x^{\mu}}.
	\end{align}
\end{prop}
\begin{proof}
	無限小並進変換
	\begin{align}
		x^{\mu}\longrightarrow{}x^{\mu}+\varepsilon^{\mu}
	\end{align}
	を考える.微小変換部分
	\begin{align}
		\hat{\varepsilon}:x^{\mu}\longrightarrow{}(x^{\mu}+\varepsilon^{\mu})-x^{\mu}
	\end{align}
	の\(x^{\mu}\)への作用を考えると,
	\begin{align}
		\hat{\varepsilon}x^{\nu} & =\varepsilon^{\mu}\delta^{\mu}_{\ \nu}                                   \\
		                         & =\left(\varepsilon^{\mu}\frac{\partial}{\partial{}x^{\mu}}\right)x^{\nu}
	\end{align}
	従って,
	\begin{align}
		\hat{\varepsilon}=\varepsilon^{\mu}\frac{\partial}{\partial{}x^{\mu}}
	\end{align}
	で表せる.従って,
	\begin{align}
		\hat{P}_{\mu}\coloneqq{}i\frac{\partial}{\partial{}x^{\mu}}
	\end{align}
	とすると,
	\begin{align}
		\hat{\varepsilon}=-i\varepsilon^{\mu}\hat{P}_{\mu}
	\end{align}
\end{proof}
\begin{prop}
	無限小並進変換のLie代数の交換関係は,
	\begin{align}
		[\hat{P}_{\mu},\hat{P}_{\nu}]=0.
	\end{align}
\end{prop}
\begin{proof}

\end{proof}
\begin{prop}[Poincar\'{e}代数の交換関係]
	\begin{align}
		[\hat{M}_{\mu\nu},\hat{P}_{\rho}]=i\eta_{\mu\rho}\hat{P}_{\nu}-i\eta_{\nu\rho}\hat{P}_{\mu}
	\end{align}
\end{prop}
\begin{proof}

\end{proof}
\section{スケール演算子}
\begin{prop}[スケール変換]
	\(1+\varepsilon\)倍するスケール変換 (scale transformation)
	\begin{align}
		x^{\mu}\longrightarrow{}(1+\varepsilon)x^{\mu}
	\end{align}
	のLie代数は
	\begin{align}
		\hat{D}\coloneqq{}ix^{\mu}\frac{\partial}{\partial{}x^{\mu}}.
	\end{align}
\end{prop}
\begin{proof}
	スケール変換は
	\begin{align}
		x^{\mu}\longrightarrow{}x^{\mu}+\varepsilon{}x^{\mu}
	\end{align}
	で,微小変換部分
	\begin{align}
		\hat{\varepsilon}:x^{\mu}\mapsto{}(x^{\mu}+\varepsilon{}x^{\mu})-x^{\mu}
	\end{align}
	の\(x^{\nu}\)への作用を考えると,
	\begin{align}
		\hat{\varepsilon}x^{\nu} & =\varepsilon{}x^{\mu}\delta^{\nu}_{\mu}                                      \\
		                         & =\left(\varepsilon{}x^{\mu}\frac{\partial}{\partial{}x^{\mu}}\right)x^{\nu}.
	\end{align}
	従って,
	\begin{align}
		\hat{D}\coloneqq{}ix^{\mu}\frac{\partial}{\partial{}x^{\mu}}
	\end{align}
	を定義すると,微小スケール変換は
	\begin{align}
		\hat{\varepsilon}=-i\varepsilon{}\hat{D}
	\end{align}
	で表される.
\end{proof}
\section{反転変換}
\begin{defn}[反転変換]
	座標変換
	\begin{align}
		\hat{I}:x^{\mu}\longrightarrow{}\frac{x^{\mu}}{x^{2}}
	\end{align}
	を反転変換という.なお,反転変換は原点に対して定義されない.
\end{defn}
\begin{prop}
	\begin{align}
		\hat{I}^{2}=1.
	\end{align}
\end{prop}
\begin{proof}
	\begin{align}
		\hat{I}^{2}x^{\mu} & =\hat{I}\frac{x^{\mu}}{x^{2}}        \\
		                   & =\frac{x^{2}}{x^{2}}x^{\mu}=x^{\mu}.
	\end{align}
\end{proof}
\begin{lem}
	\begin{align}
		\frac{\partial}{\partial{}x^{\nu}}\frac{x^{\mu}}{x^{2}}=\frac{1}{x^{2}}\left(\delta^{\mu}_{\ \nu}+\frac{2x^{\mu}x_{\nu}}{x^{2}}\right)
	\end{align}
\end{lem}
\begin{proof}
	\begin{align}
		\frac{\partial}{\partial{}x^{\nu}}\frac{x^{\mu}}{x^{2}} & =\frac{\partial{}x^{\mu}}{\partial{}x^{\nu}}\frac{1}{x^{2}}+x^{\mu}\frac{\partial}{\partial{}x^{\nu}}\frac{1}{x^{2}} \\
		                                                        & =\delta^{\mu}_{\ \nu}\frac{1}{x^{2}}-x^{\mu}\frac{2x_{\nu}}{(x^{2})^{2}}                                             \\
		                                                        & =\frac{1}{x^{2}}\left(\delta^{\mu}_{\ \nu}-\frac{2x^{\mu}x_{\nu}}{x^{2}}\right)
	\end{align}
\end{proof}
\begin{prop}
	\begin{align}
		\hat{I}\hat{D}\hat{I}=-\hat{D}.
	\end{align}
\end{prop}
\begin{proof}
	まず,
	\begin{align}
		\hat{D}\hat{I}f(x) & =ix^{\mu}\frac{\partial}{\partial{}x^{\mu}}f(y),\quad{}y\coloneqq{}\frac{x^{\mu}}{x^{2}}                                        \\
		                   & =ix^{\mu}\frac{\partial{}y^{\nu}}{\partial{}x^{\mu}}\frac{\partial{}f(y)}{\partial{}y^{\nu}}                                    \\
		                   & =ix^{\mu}\frac{1}{x^{2}}\left(\delta^{\nu}_{\ \mu}-\frac{2x^{\nu}x_{\mu}}{x^{2}}\right)\frac{\partial{}f(y)}{\partial{}y^{\nu}} \\
		                   & =i\frac{1}{x^{2}}\left(x^{\nu}-\frac{2x^{2}x^{\nu}}{x^{2}}\right)\frac{\partial{}f(y)}{\partial{}y^{\nu}}                       \\
		                   & =-i\frac{x^{\nu}}{x^{2}}\frac{\partial{}f(y)}{\partial{}y^{\nu}}=-iy^{\nu}\frac{\partial{}f(y)}{\partial{}y^{\nu}}
	\end{align}
	従って,
	\begin{align}
		\hat{I}\hat{D}\hat{I}f(x) & =-ix^{\nu}\frac{\partial{}f(x)}{\partial{}x^{\nu}} \\
		                          & =-\hat{D}f(x).
	\end{align}
\end{proof}
\section{特殊共形変換}
\begin{defn}[特殊共形変換]
	\begin{align}
		\hat{K}_{\mu}\coloneqq{}\hat{I}\hat{P}_{\mu}\hat{I}
	\end{align}
	を特殊共形変換演算子として定義する.
\end{defn}
\begin{prop}
	特殊共形変換演算子は
	\begin{align}
		\hat{K}_{\mu}=i\left(x^{2}\delta^{\nu}_{\ \mu}+2x^{\nu}x_{\mu}\right)\frac{\partial{}}{\partial{}x^{\nu}}
	\end{align}
	で表せる.
\end{prop}
\begin{proof}
	まず,
	\begin{align}
		\hat{P}_{\mu}\hat{I}f(x) & =\hat{P}_{\mu}f(y),\quad{}y^{\mu}=\frac{x^{\mu}}{x^{2}}                                                                  \\
		                         & =i\frac{\partial{}y^{\nu}}{\partial{}x^{\mu}}\frac{\partial{}f(y)}{\partial{}y^{\nu}}                                    \\
		                         & =i\frac{1}{x^{2}}\left(\delta^{\nu}_{\ \mu}+\frac{2x^{\nu}x_{\mu}}{x^{2}}\right)\frac{\partial{}f(y)}{\partial{}y^{\nu}} \\
		                         & =i\frac{1}{x^{2}}\left(\delta^{\nu}_{\ \mu}+2y^{\nu}x_{\mu}\right)\frac{\partial{}f(y)}{\partial{}y^{\nu}}
	\end{align}
	ここで,
	\begin{align}
		x^{2}=\frac{1}{y^{2}}
	\end{align}
	であることを用いると,
	\begin{align}
		\hat{P}_{\mu}\hat{I}f(x)=i\left(y^{2}\delta^{\nu}_{\ \mu}+2y^{\nu}y_{\mu}\right)\frac{\partial{}f(y)}{\partial{}y^{\nu}}.
	\end{align}
	従って,
	\begin{align}
		\hat{I}\hat{P}_{\mu}\hat{I}f(x)=i\left(x^{2}\delta^{\nu}_{\ \mu}+2x^{\nu}x_{\mu}\right)\frac{\partial{}}{\partial{}x^{\nu}}f(y)
	\end{align}
\end{proof}
\section{共形変換代数}
\begin{defn}[共形変換代数]
	Poincar\'{e}代数\(\hat{M}_{\mu\nu}\),\(\hat{P}_{\mu}\),スケール演算子
	\(\hat{D}\),特殊共形変換演算子\(\hat{K}_{\mu}\)が成すLie代数を
	共形変換代数という.これを\(\mathfrak{c}(3,1)\)で表す.
\end{defn}
\begin{prop}
	\begin{align}
		\mathfrak{c}(3,1)\simeq{}\mathfrak{so}(4,2)
	\end{align}
\end{prop}
\begin{proof}
	\(0\leq{}\mu,\nu\leq{}3\)として,
	\begin{align}
		M_{4\ -1}   & \coloneqq{}\hat{D},                                  \\
		M_{\mu\ -1} & \coloneqq{}\frac{1}{2}(\hat{K}_{\mu}+\hat{P}_{\mu}), \\
		M_{\mu\ 4}  & \coloneqq{}\frac{1}{2}(\hat{K}_{\mu}-\hat{P}_{\mu}).
	\end{align}
	として,計量
	\begin{align}
		\eta\coloneqq{}\begin{bmatrix}
			-1 &                     \\
			   & -1 &                \\
			   &    & 1              \\
			   &    &   & \ddots     \\
			   &    &   &        & 1
		\end{bmatrix}
	\end{align}
	を導入すると,
	\begin{align}
		[\hat{M}_{\mu\nu},\hat{M}_{\rho\sigma}]=i\eta_{\mu\rho}\hat{M}_{\nu\sigma}-i\eta_{\nu\rho}\hat{M}_{\mu\sigma}-i\eta_{\mu\sigma}\hat{M}_{\nu\rho}+i\eta_{\nu\sigma}\hat{M}_{\mu\rho}.
	\end{align}
\end{proof}
\section{2次元の共形変換}
\chapter{場と共形不変性}
\section{場の共形変換}
\(d\)次元時空における\(N\)成分場\(\phi^{a}(x)\)の共形変換を考える.
\bibliography{bibliography}
\bibliographystyle{junsrt}
\end{document}