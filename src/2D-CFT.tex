\chapter{2次元共形場理論}
\section{2次元の共形変換}
\(g_{\mu\nu}=\delta_{\mu\nu}\)である2次元ユークリッド空間の共形変換を考える.すると,
\cref{eq::conformal-Killing-equation-2}より,
\begin{align}
	\partial_{\mu}\varepsilon_{\nu}(x)+\partial_{\nu}\varepsilon_{\mu}(x)=\frac{2}{d}\delta_{\mu\nu}\partial_{\rho}\varepsilon^{\rho}(x).
\end{align}
これは,
\begin{align}
	\frac{\partial{}\varepsilon_{1}(x)}{\partial{}x^{0}}+\frac{\partial{}\varepsilon_{0}(x)}{\partial{}x^{1}} & =0,                                                                                                         \\
	\frac{\partial{}\varepsilon_{1}(x)}{\partial{}x^{1}}+\frac{\partial{}\varepsilon_{1}(x)}{\partial{}x^{1}} & =\frac{\partial{}\varepsilon_{0}(x)}{\partial{}x^{0}}+\frac{\partial{}\varepsilon_{1}(x)}{\partial{}x^{1}}, \\
	\frac{\partial{}\varepsilon_{0}(x)}{\partial{}x^{0}}+\frac{\partial{}\varepsilon_{0}(x)}{\partial{}x^{0}} & =\frac{\partial{}\varepsilon_{0}(x)}{\partial{}x^{0}}+\frac{\partial{}\varepsilon_{1}(x)}{\partial{}x^{1}}.
\end{align}
と同値である.これを整理すると,Cauchy--Riemann関係式
\begin{align}
	\frac{\partial{}\varepsilon_{1}(x)}{\partial{}x^{0}}+\frac{\partial{}\varepsilon_{0}(x)}{\partial{}x^{1}} & =0, \\
	\frac{\partial{}\varepsilon_{0}(x)}{\partial{}x^{0}}-\frac{\partial{}\varepsilon_{1}(x)}{\partial{}x^{1}} & =0.
\end{align}
に帰着する.従って,複素座標\(z\coloneqq{}x_{0}+ix_{1}\)と複素数場
\begin{align}
	\varepsilon(z)\coloneqq{}\varepsilon_{0}(z)+i\varepsilon_{1}(z)
\end{align}
を導入すると,共形Killing方程式は
\begin{align}
	\frac{\partial\varepsilon(z)}{\partial{}z}=0
\end{align}
と同値.
